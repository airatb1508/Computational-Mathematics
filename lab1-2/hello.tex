\hypertarget{ux446ux435ux43bux44c-ux440ux430ux431ux43eux442ux44b}{%
\subsection{Цель
работы:}\label{ux446ux435ux43bux44c-ux440ux430ux431ux43eux442ux44b}}

Научиться решать нелинейные уравнения методом простых итераций, методом
Ньютона и модифицированным методом Ньютона с помощью ЭВМ.

\hypertarget{ux441ux43eux434ux435ux440ux436ux430ux43dux438ux435-ux440ux430ux431ux43eux442ux44b}{%
\subsection{Содержание
работы:}\label{ux441ux43eux434ux435ux440ux436ux430ux43dux438ux435-ux440ux430ux431ux43eux442ux44b}}

\begin{enumerate}
\def\labelenumi{\arabic{enumi}.}
\item
  Изучить метод простых итераций, метод Ньютона и модифицированный метод
  Ньютона для решения нелинейных уравнений.
\item
  На конкретном примере усвоить порядок решения нелинейных уравнений с
  помощью ЭВМ указанными методами.
\item
  Составить программу (программы) на любом языке программирования и с ее
  помощью решить уравнение с точностью \(\varepsilon = 0.001\) и
  \(\delta = 0.01\). Сделать вывод о скорости сходимости всех трех
  методов.
\item
  Изменить \(\varepsilon = \varepsilon / 10, \delta = \delta / 10\) и
  снова решить задачу. Сделать выводы о: скорости сходимости
  рассматриваемых методов; влиянии точности на скорость сходимости;
  влиянии выбора начального приближения в методе простых итераций на
  скорость сходимости.
\end{enumerate}

\hypertarget{ux437ux430ux434ux430ux43dux438ux435}{%
\subsection{Задание:}\label{ux437ux430ux434ux430ux43dux438ux435}}

\begin{enumerate}
\def\labelenumi{\arabic{enumi}.}
\item
  Доказать графическим и аналитическим методами существование
  единственного корня нелинейного уравнения \begin{align}\label{eq1}
  f(x) = 3 \cos(2x) - x + 0.25
  \end{align} на отрезке \([3.2, 3.5]\).
\item
  Построить рабочие формулы метода простых итераций, метода Ньютона и
  модифицированного метода Ньютона, реализующие процесс поиска корня
  нелинейного уравнения (\ref{eq1}) на указанном отрезке.
\item
  Составить программу (программы) на любом языке программирования,
  реализующие построенные итерационные процессы.
\end{enumerate}

\hypertarget{ux440ux435ux448ux435ux43dux438ux435}{%
\subsection{Решение:}\label{ux440ux435ux448ux435ux43dux438ux435}}

\begin{enumerate}
\def\labelenumi{\arabic{enumi}.}
\item
  Докажем \textbf{графическим} методом единственность корня нелинейного
  уравнения (\ref{eq1}). Из графика функции
  \(f(x) = 3 \cos(2x) - x + 0.25 = 0\) на Рис.1 видно, что функция
  \(f(x)\) пересекает ось в одной точке, являющейся приближенным
  значением корня нелинейного уравнения (\ref{eq1}). Но так как данная
  функция имеет сложный аналитический вид, то преобразуем уравнение
  (\ref{eq1}) к виду \(\cos(2x) = x - 0.25\) и построим два графика
  \(y = \cos(2x)\) и \(y = x - 0.25\), имеющих более простой
  аналитический вид (Рис.2). Абсцисса точки пересечения графиков
  является приближенным значением корня. Заметим, что графический метод
  показывает количество корней исходного уравнения, но не доказывает
  единственность корня на отрезке.

  \vspace{1em}

  \begin{tikzpicture}[scale=0.77]
      \begin{scope}
          \label{gr1}
          \begin{axis}[
              title={Рисунок 1},
              domain=3.2:3.5,
              axis lines = middle,
              ymax=0.1,
              ymin=-0.3,
              xlabel={$x$},
              ylabel={$f(x)$},
              legend pos={south west},
          ]
              \addplot[samples=120] {3*cos(deg(2*x))-x+0.25};
              \addlegendentry{$3 \cos(2x) - x + 0.25$};
          \end{axis}
      \end{scope}
      \begin{scope}[xshift=9cm]
          \label{gr2}
          \begin{axis}[
              title={Рисунок 2},
              domain=0:5,
              axis lines=middle,
              ymax=5,
              ymin=0,
              xlabel={$x$},
              ylabel={$y$},
              xtick=\empty,
              ytick=\empty,
              legend pos={south west},
          ]
              \addplot[solid,samples=50] {3*cos(deg(2*x))};
              \addlegendentry{$3 \cos(2x)$};
              \addplot[dashed,samples=50] {x-0.25};
              \addlegendentry{$x - 0.25$};
          \end{axis}
      \end{scope}
  \end{tikzpicture}

  \textbf{Аналитический метод.} Функция \(f(x)\) непрерывна на отрезке
  {[}3.2, 3.5{]}, имеет на концах отрезка разные знаки
  \((f(3.2) = 0.029555; f(3.5) = -0.98829)\), а производная функции
  \(f(x)\) не меняет знак на отрезке
  \((f'(x) = -6 \sin(2x) - 1) > 0, \forall x \in [3.2, 3.5])\).
  Следовательно, нелинейное уравнение (\ref{eq1}) имеет на указанном
  отрезке единственный корень.
\item
  \textbf{Метод простых итераций.} Построим функцию
  \(\phi (x) = x + cf(x)\). Константа \(c\) выбирается из достаточного
  условия сходимости

  \begin{align}\label{eq2}
  \left | \phi '(x) \right | < 1, \forall x \in [a, b]
  \end{align}

  Если производная \(f '(x) > 0, \forall x \in [a, b]\), то значение
  \(c\) выбирается из интервала \(\frac{-2}{f'(x)} < c < 0\), если
  производная \(f '(x) < 0, \forall x \in [a, b]\), то -- из интервала
  \(0 < c < \frac{-2}{f'(x)}\). Так как для рассматриваемого примера
  \(0 < c < \frac{-2}{f'(x)}\) всюду отрицательна на отрезке
  \([3.2, 3.5]\), то придавая переменной \(x\) различные значения из
  интервала \([3.2, 3.5]\) и выбирая наименьший интервал
  \(\frac{-2}{f'(x)} < c < 0\), получим \(0 < c < 0.4047\). Выбираем
  произвольное значение \(c\) из этого интервала. Пусть \(c = 0.2\).
  Тогда рабочая формула метода простых итераций будет иметь вид:

  \begin{align}\label{eq3}
  x_{n+1} = x_n + 0.2 \times (3 \cos(2 x_n) - x_n + 0.25), n = 0, 1, 2, ...
  \end{align}

  Итерационный процесс (\ref{eq3}) можно начать, задав произвольное
  начальное приближение \(x_0 = [3.2, 3.5]\). Итерационный процесс
  (\ref{eq3}) заканчивается при одновременном выполнении двух условий:

  \begin{align}\label{eq4}
  \left | x_{n+1} - x_n \right | \leq \varepsilon \text{ и } \left | f(x_n+1) \leq \delta \right |
  \end{align}

  В этом случае значение является приближенным значением корня
  нелинейного уравнения (\ref{eq1}) на отрезке \([3.2, 3.5]\).

  \textbf{Метод Ньютона.} В качестве начального приближения \(x_0\)
  здесь выбирается правый или левый конец отрезка, в зависимости от
  того, в котором выполняется достаточное условие сходимости метода
  Ньютона вида:

  \begin{align}\label{eq5}
  f(x_0) f''(x_0) > 0
  \end{align}

  Заметим, что в точке \(x = 3.2\) условие (\ref{eq5}) не выполняется, а
  в точке \(x = 3.5\) -- выполняется. Следовательно, в качестве
  начального приближения выбирается точка \(x_0 = 3.5\). Рабочая формула
  метода Ньютона
  \(x_{n+1} = x_n - \frac{f(x_n)}{f'(x_n)}, n = 0, 1, 2, ...\) для
  данного уравнения запишется так:

  \begin{align}\label{eq6}
  x_{n+1} = x_n - \frac{3 \cos(2x) - x + 0.25}{-6 \sin(2x) - 1)}, n = 0, 1, 2, ...
  \end{align}

  Условия выхода итерационного процесса (\ref{eq6}) аналогичны условиям
  (\ref{eq4}) метода простых итераций.

  \textbf{Модифицированный метод Ньютона.} Начальное приближение
  выбирается аналогично методу Ньютона, т.е. \(x_0 = 0\). Рабочая
  формула модифицированного метода Ньютона для данного примера запишется
  так:

  \begin{align}\label{eq7}
  x_{n+1} = x_n - \frac{3 \cos(2x) - x + 0.25}{-6 \sin(2x) - 1)}, n = 0, 1, 2, ...
  \end{align}

  Условия выхода итерационного процесса (\ref{eq7}) аналогичны условиям
  (\ref{eq4}) метода простых итераций.

  \emph{Замечание:} для того, чтобы сделать вывод о скорости сходимости
  методов, необходимо в каждом методе выбирать одинаковое начальное
  приближение.
\item
  \textbf{Блок-схема метода простых итераций}, метода Ньютона и
  модифицированного метода Ньютона приведена на рисунке 3.
\end{enumerate}

\begin{center}
\begin{tikzpicture}[node distance = 1cm, scale=0.8, every node/.style={scale=0.8}]
    \label{gr3}
    \node [block] (bl1) {1. Задать параметры метода};
    \node [block,below of=bl1,node distance=2.2cm] (bl2) {2. Вычислить очередное приближение.};
    \node [block,below of=bl2,node distance=2.2cm] (bl3) {3. Проверить условия окончания процесса.};
    \node [block,left of=bl3,  node distance=8cm] (bl41) {4. Обновить начальное приближение.};
    \node [block,below of=bl3,node distance=3.2cm] (bl42) {5. Распечатать приближенное значение корня.};
    \node [block,below of=bl42,node distance=2.2cm] (bl5) {6. Остановка.};

    \path [line] (bl1) -- (bl2);
    \path [line] (bl2) -- (bl3);
    \path [line] (bl3) -- node [above=0.1em] {нет} (bl41);
    \path [line] (bl3) -- node [left=0.2em] {да} (bl42);
    \path [line] (bl41) |- (bl2);
    \path [line] (bl42) -- (bl5);
\end{tikzpicture}
\end{center}
\begin{center}
Рисунок 3.
\end{center}

\hypertarget{ux43fux440ux43eux433ux440ux430ux43cux43cux430}{%
\subsection{Программа}\label{ux43fux440ux43eux433ux440ux430ux43cux43cux430}}

\begin{verbatim}
#include <stdio.h>
#include <math.h>

int main() {
    int   n = 0;
    float x0 = -1;
    float c = -0.1;
    float eps = 0.001;
    float d = 0.01;
    float x = x0;
    float y;
    float z;
    clrscr();
    do {
        y = x + c * (exp(x) + x);
        z = x;
        printf("%d %.4f %.4f %.4f %.4f\n", n++, x, y, fabs(y - x),
        fabs(exp(y) + y));
        x = y;
    } while(fabs(z - x) > e || fabs(exp(x) + x) > d;
    return (0);
}
\end{verbatim}
