\documentclass[utf8x, 14pt, oneside, a4paper]{article}
%%%
%%% -- author: https://github.com/Amet13/master-thesis
%%%
%%% Преамбула %%%


\usepackage{fontspec} % XeTeX
\usepackage{xunicode} % Unicode для XeTeX
\usepackage{xltxtra}  % Верхние и нижние индексы
\usepackage{pdfpages} % Вставка PDF
\usepackage{extsizes} % Для добавления в параметры класса документа 14pt
\usepackage{tikz}
\usepackage{pgfplots}
\pgfplotsset{compat=1.9}

\usepackage{listings} % Оформление исходного кода
\lstset{
    basicstyle=\footnotesize\ttfamily, % Размер и тип шрифта
    breaklines=true,                   % Перенос строк
    tabsize=4,                         % Размер табуляции
    literate={--}{{-{}-}}2,            % Корректно отображать двойной дефис
    literate={---}{{-{}-{}-}}3         % Корректно отображать тройной дефис
}

% Шрифты, xelatex
\defaultfontfeatures{Ligatures=TeX}
\setmainfont{Times New Roman} % Нормоконтроллеры хотят именно его
\newfontfamily\cyrillicfont{Times New Roman}
% \setsansfont{Liberation Sans} % Тут я его не использую, но если пригодится
\setmonofont{FreeMono} % Моноширинный шрифт для оформления кода

% Формулы
\usepackage{amsmath, amsfonts}
\usepackage{mathtools,unicode-math} % Не совместим с amsmath
\setmathfont{XITS Math}             % Шрифт для формул: https://github.com/khaledhosny/xits-math
%\numberwithin{equation}{section}    % Формула вида секция.номер

% Русский язык
\usepackage{polyglossia}
\setdefaultlanguage{russian}

% Абзацы и списки
\usepackage{enumerate}   % Тонкая настройка списков
\usepackage{indentfirst} % Красная строка после заголовка
\usepackage{float}       % Расширенное управление плавающими объектами
\usepackage{multirow}    % Сложные таблицы
\renewcommand{\baselinestretch}{1.5} % Полуторный межстрочный интервал
%\parindent 1.25cm % Абзацный отступ
\setlength\parindent{1.25cm}
% \setlength{\parskip}{0.5em}

\usepackage{textcase}

% Пути к каталогам с изображениями
\usepackage{graphicx} % Вставка картинок и дополнений
\graphicspath{{assets/}}

% Формат подрисуночных записей
\usepackage{chngcntr}

% Сбрасываем счетчик таблиц и рисунков в каждой новой главе
\counterwithin{figure}{section}
\counterwithin{table}{section}

% Гиперссылки
\usepackage{hyperref}
\hypersetup{
    colorlinks, urlcolor={black}, % Все ссылки черного цвета, кликабельные
    linkcolor={black}, citecolor={black}, filecolor={black},
}

% Оформление библиографии и подрисуночных записей через точку
\makeatletter
\renewcommand*{\@biblabel}[1]{\hfill#1.}
\renewcommand*\l@section{\@dottedtocline{1}{1em}{1em}}
\renewcommand{\thefigure}{\thesection.\arabic{figure}} % Формат рисунка секция.номер
\renewcommand{\thetable}{\thesection.\arabic{table}}   % Формат таблицы секция.номер
\def\redeflsection{\def\l@section{\@dottedtocline{1}{0em}{10em}}}
\makeatother

\sloppy             % Избавляемся от переполнений
\hyphenpenalty=1000 % Частота переносов
\clubpenalty=10000  % Запрещаем разрыв страницы после первой строки абзаца
\widowpenalty=10000 % Запрещаем разрыв страницы после последней строки абзаца

% Отступы у страниц
% \usepackage[left=30mm,right=15mm,top=20mm,bottom=20mm]{geometry}

% Списки
\usepackage{enumitem}
\setlist[enumerate,itemize]{leftmargin=12.5mm} % Отступы в списках

\makeatletter
    \AddEnumerateCounter{\asbuk}{\@asbuk}{м)}
\makeatother
\setlist{nolistsep}                           % Нет отступов между пунктами списка
\renewcommand{\labelitemi}{—}                % Маркер списка —
% \renewcommand{\labelenumi}{\asbuk{enumi})}    % Список второго ровня
\renewcommand{\labelenumii}{\asbuk{enumii})} % Список третьего урвня

% Содержание
\usepackage{tocloft}
\renewcommand{\cfttoctitlefont}{\hspace{0.38\textwidth}\bfseries\MakeTextUppercase} % СОДЕРЖАНИЕ
\renewcommand{\cftsecfont}{\hspace{0pt}}            % Имена секций в содержании не жирным шрифтом
\renewcommand\cftsecleader{\cftdotfill{\cftdotsep}} % Точки для секций в содержании
\renewcommand\cftsecpagefont{\mdseries}             % Номера страниц не жирные
\setcounter{tocdepth}{1}                            % Глубина оглавления, до subsection

% Список иллюстративного материала
\renewcommand{\cftloftitlefont}{\hspace{0.17\textwidth}\MakeTextUppercase}
\renewcommand{\cftfigfont}{Рисунок }
\addto\captionsrussian{\renewcommand\listfigurename{Список иллюстративного материала}}

% Список табличного материала
\renewcommand{\cftlottitlefont}{\hspace{0.2\textwidth}\MakeTextUppercase}
\renewcommand{\cfttabfont}{Таблица }
\addto\captionsrussian{\renewcommand\listtablename{Список табличного материала}}

% Формат подрисуночных надписей
\RequirePackage{caption}
\DeclareCaptionLabelSeparator{defffis}{ -- } % Разделитель
\captionsetup[figure]{justification=centering, labelsep=defffis, format=plain} % Подпись рисунка по центру
\captionsetup[table]{justification=raggedright, labelsep=defffis, format=plain, singlelinecheck=false} % Подпись таблицы слева
\addto\captionsrussian{\renewcommand{\figurename}{Рисунок}} % Имя фигуры

% Пользовательские функции
\newcommand{\addimg}[4]{ % Добавление одного рисунка
    \begin{figure}
        \centering
        \includegraphics[width=#2\linewidth]{#1}
        \caption{#3} \label{#4}
    \end{figure}
}
\newcommand{\addimghere}[4]{ % Добавить рисунок непосредственно в это место
    \begin{figure}[H]
        \centering
        \includegraphics[width=#2\linewidth]{#1}
        \caption{#3} \label{#4}
    \end{figure}
}
\newcommand{\addtwoimghere}[5]{ % Вставка двух рисунков
    \begin{figure}[H]
        \centering
        \includegraphics[width=#2\linewidth]{#1}
        \hfill
        \includegraphics[width=#3\linewidth]{#2}
        \caption{#4} \label{#5}
    \end{figure}
}

% Секции без номеров (введение, заключение...), вместо section*{}
\newcommand{\anonsection}[1]{
    \phantomsection % Корректный переход по ссылкам в содержании
    \paragraph{\centerline{\MakeTextUppercase{{#1}}}\vspace{1em}}
    \addcontentsline{toc}{section}{#1}
}

% toc uppercase
\usepackage{etoolbox}
\makeatletter
\patchcmd{\l@section}{#1}{\MakeUppercase{#1}}{}{}
\patchcmd{\l@subsection}{#1}{\MakeUppercase{#1}}{}{}
\makeatother

% Секция для аннотации (она не включается в содержание)
\newcommand{\annotation}[1]{
    \paragraph{\centerline{{#1}}\vspace{1em}}
}

% Секция для списка иллюстративного материала
\newcommand{\lof}{
    \phantomsection
    \listoffigure
    \addcontentsline{toc}{section}{\listfigurename}
}

% Секция для списка табличного материала
\newcommand{\lot}{
    \phantomsection
    \listoftables
    \addcontentsline{toc}{section}{\listtablename}
}

% Секции для приложений
\newcommand{\appsection}[1]{
    \phantomsection
    \paragraph{\centerline{{#1}}}
    \addcontentsline{toc}{section}{{#1}}
}

% Библиография: отступы и межстрочный интервал
\makeatletter
\renewenvironment{thebibliography}[1]
    {\section*{\refname}
        \list{\@biblabel{\@arabic\c@enumiv}}
           {\settowidth\labelwidth{\@biblabel{#1}}
            \leftmargin\labelsep
            \itemindent 16.7mm
            \@openbib@code
            \usecounter{enumiv}
            \let\p@enumiv\@empty
            \renewcommand\theenumiv{\@arabic\c@enumiv}
        }
        \setlength{\itemsep}{0pt}
    }
\makeatother

\usepackage[nottoc,numbib]{tocbibind}	% добавить в toc и нумеровать

% Переопределение стандартных \section, \subsection, \subsubsection по ГОСТу;
% Переопределение их отступов до и после для 1.5 интервала во всем документе
\usepackage{titlesec}

\titleformat{\section}[block]
{\bfseries\normalsize}{\thesection}{1em}{\MakeTextUppercase}
\titlespacing\section{\parindent}{\parskip}{\parskip}

\titleformat{\subsection}[hang]
{\bfseries\normalsize}{\thesubsection}{1em}{}
\titlespacing\subsection{\parindent}{\parskip}{\parskip}

\titleformat{\subsubsection}[hang]
{\bfseries\normalsize}{\thesubsubsection}{1em}{}
\titlespacing\subsubsection{\parindent}{\parskip}{\parskip}


\usepackage{lastpage} % Подсчет количества страниц
\setcounter{page}{1}  % Начало нумерации страниц

\usetikzlibrary{shapes, arrows}
\tikzstyle{decision} = [diamond, draw, fill=white, 
    text width=4.5em, text badly centered, node distance=3cm, inner sep=0pt]
\tikzstyle{block} = [rectangle, draw, fill=white, 
    text width=12em, text centered, minimum height=2em]
\tikzstyle{line} = [draw, -latex']
\tikzstyle{cloud} = [draw, ellipse,fill=red!20, node distance=3cm,
    minimum height=2em]


\title{Вычислительная математика\\
	\large Лабораторная работа №1-2\\
	Вариант №14}
\author{Бадрутдинов Айрат 4302}

\begin{document}
\maketitle
\thispagestyle{empty}
\clearpage
\tableofcontents
\clearpage

\section{Цель работы:}
Научиться решать нелинейные уравнения методом простых итераций,
методом Ньютона и модифицированным методом Ньютона с помощью ЭВМ.

\section{Содержание работы:}

1. Изучить метод простых итераций, метод Ньютона и модифицированный метод
   Ньютона для решения нелинейных уравнений.

2. На конкретном примере усвоить порядок решения нелинейных уравнений с помощью
   ЭВМ указанными методами.

3. Составить программу (программы) на любом языке программирования и с ее
   помощью решить уравнение с точностью $\varepsilon = 0.001$ и 
   $\delta = 0.01$. Сделать вывод о скорости сходимости всех трех методов.

4. Изменить $\varepsilon = \varepsilon / 10, \delta = \delta / 10$ и снова
   решить задачу. Сделать выводы о: скорости сходимости рассматриваемых
   методов; влиянии точности на скорость сходимости; влиянии выбора начального
   приближения в методе простых итераций на скорость сходимости.

\section{Задание (Вариант №14)}

1. Доказать графическим и аналитическим методами существование единственного
   корня нелинейного уравнения
   \begin{align}\label{eq1}
   f(x) = 3 \cos(2x) - x + 0.25
   \end{align}
   на отрезке $[3.2, 3.5]$.

2. Построить рабочие формулы метода простых итераций, метода Ньютона и
   модифицированного метода Ньютона, реализующие процесс поиска корня
   нелинейного уравнения (\ref{eq1}) на указанном отрезке.

3. Составить программу (программы) на любом языке программирования, реализующие
   построенные итерационные процессы.


\section{Математическая часть работы}

1.  Докажем \underline{графическим методом} единственность корня нелинейного уравнения
	(\ref{eq1}). Из графика функции
	$f(x) = 3 \cos(2x) - x + 0.25 = 0$
	на Рис.1 видно, что функция $f(x)$ пересекает ось в одной точке, являющейся
	приближенным значением корня нелинейного уравнения (\ref{eq1}). Но так как
	данная функция имеет сложный аналитический вид, то преобразуем уравнение
	(\ref{eq1}) к виду $\cos(2x) = x - 0.25$ и построим два графика $y =
	\cos(2x)$ и $y = x - 0.25$, имеющих более простой аналитический вид
	(Рис.2). Абсцисса точки пересечения графиков является приближенным
	значением корня. Заметим, что графический метод показывает количество
	корней исходного уравнения, но не доказывает единственность корня на
	отрезке.

	\vspace{1em}

	\begin{tikzpicture}[scale=0.77]
		\begin{scope}
			\label{gr1}
			\begin{axis}[
				title={Рисунок 1},
				domain=3.2:3.5,
				axis lines = middle,
				ymax=0.1,
				ymin=-0.3,
				xlabel={$x$},
				ylabel={$f(x)$},
				legend pos={south west},
			]
				\addplot[samples=120] {3*cos(deg(2*x))-x+0.25};
				\addlegendentry{$3 \cos(2x) - x + 0.25$};
			\end{axis}
		\end{scope}
		\begin{scope}[xshift=9cm]
			\label{gr2}
			\begin{axis}[
				title={Рисунок 2},
				domain=0:5,
				axis lines=middle,
				ymax=5,
				ymin=0,
				xlabel={$x$},
				ylabel={$y$},
				xtick=\empty,
				ytick=\empty,
				legend pos={south west},
			]
				\addplot[solid,samples=50] {3*cos(deg(2*x))};
				\addlegendentry{$3 \cos(2x)$};
				\addplot[dashed,samples=50] {x-0.25};
				\addlegendentry{$x - 0.25$};
			\end{axis}
		\end{scope}
	\end{tikzpicture}

	\underline{Аналитический метод.} Функция $f(x)$ непрерывна на отрезке [3.2, 3.5],
	имеет на концах отрезка разные знаки $(f(3.2) = 0.029555; f(3.5) =
	-0.98829)$, а производная функции $f(x)$ не меняет знак на отрезке $(f'(x)
	= -6 \sin(2x) - 1) > 0, \forall x \in [3.2, 3.5])$. Следовательно, нелинейное
	уравнение (\ref{eq1}) имеет на указанном отрезке единственный корень.

2.  \underline{Метод простых итераций.}
	Построим функцию $\phi (x) = x + cf(x)$. Константа $c$ выбирается из
	достаточного условия сходимости

	\begin{align}\label{eq2}
	\left | \phi '(x) \right | < 1, \forall x \in [a, b]
	\end{align}

	Если производная $f '(x) > 0, \forall x \in [a, b]$, то значение $c$
	выбирается из интервала $\frac{-2}{f'(x)} < c < 0$, если производная $f '(x) <
	0, \forall x \in [a, b]$, то – из интервала $0 < c < \frac{-2}{f'(x)}$. Так как
	для рассматриваемого примера $0 < c < \frac{-2}{f'(x)}$ всюду отрицательна на
	отрезке $[3.2, 3.5]$, то придавая переменной $x$ различные значения из
	интервала $[3.2, 3.5]$ и выбирая наименьший интервал $\frac{-2}{f'(x)} < c <
	0$, получим $0 < c < 0.4047$. Выбираем произвольное значение $c$ из этого
	интервала. Пусть $c = 0.2$. Тогда рабочая формула метода простых итераций будет
	иметь вид:

	\begin{align}\label{eq3}
	x_{n+1} = x_n + 0.2 \times (3 \cos(2 x_n) - x_n + 0.25), n = 0, 1, 2, ...
	\end{align}

	Итерационный процесс (\ref{eq3}) можно начать, задав произвольное начальное
	приближение $x_0 = [3.2, 3.5]$. Итерационный процесс (\ref{eq3}) заканчивается
	при одновременном выполнении двух условий:

	\begin{align}\label{eq4}
	\left | x_{n+1} - x_n \right | \leq \varepsilon \text{ и } \left | f(x_n+1) \leq \delta \right |
	\end{align}

	В этом случае значение  является приближенным значением корня нелинейного
	уравнения (\ref{eq1}) на отрезке $[3.2, 3.5]$.

	\underline{Метод Ньютона.} В качестве начального приближения $x_0$ здесь выбирается
	правый или левый конец отрезка, в зависимости от того, в котором выполняется
	достаточное условие сходимости метода Ньютона вида:

	\begin{align}\label{eq5}
	f(x_0) f''(x_0) > 0
	\end{align}

	Заметим, что в точке $x = 3.2$ условие (\ref{eq5}) не выполняется, а в точке $x
	= 3.5$ -- выполняется. Следовательно, в качестве начального приближения
	выбирается точка $x_0 = 3.5$. Рабочая формула метода Ньютона
	$x_{n+1} = x_n - \frac{f(x_n)}{f'(x_n)}, n = 0, 1, 2, ...$ для данного
	уравнения запишется так:

	\begin{align}\label{eq6}
	x_{n+1} = x_n - \frac{3 \cos(2x) - x + 0.25}{-6 \sin(2x) - 1)}, n = 0, 1, 2, ...
	\end{align}

	Условия выхода итерационного процесса (\ref{eq6}) аналогичны условиям
	(\ref{eq4}) метода простых итераций.

	\underline{Модифицированный метод Ньютона.} Начальное приближение  выбирается
	аналогично методу Ньютона, т.е. $x_0 = 0$. Рабочая формула модифицированного
	метода Ньютона  для данного примера запишется так:

	\begin{align}\label{eq7}
	x_{n+1} = x_n - \frac{3 \cos(2x) - x + 0.25}{-6 \sin(2x) - 1)}, n = 0, 1, 2, ...
	\end{align}

	Условия выхода итерационного процесса (\ref{eq7}) аналогичны условиям
	(\ref{eq4}) метода простых итераций.

	\textit{Замечание:} для того, чтобы сделать вывод о скорости сходимости методов,
	необходимо в каждом методе выбирать одинаковое начальное приближение.

3. \underline{Блок-схема метода простых итераций}, метода Ньютона и модифицированного метода Ньютона приведена на рисунке 3.

\begin{center}
\begin{tikzpicture}[node distance = 1cm, scale=0.8, every node/.style={scale=0.8}]
	\label{gr3}
	\node [block] (bl1) {1. Задать параметры метода};
	\node [block,below of=bl1,node distance=2.2cm] (bl2) {2. Вычислить очередное приближение.};
	\node [block,below of=bl2,node distance=2.2cm] (bl3) {3. Проверить условия окончания процесса.};
	\node [block,left of=bl3,  node distance=8cm] (bl41) {4. Обновить начальное приближение.};
	\node [block,below of=bl3,node distance=3.2cm] (bl42) {5. Распечатать приближенное значение корня.};
	\node [block,below of=bl42,node distance=2.2cm] (bl5) {6. Остановка.};

	\path [line] (bl1) -- (bl2);
	\path [line] (bl2) -- (bl3);
	\path [line] (bl3) -- node [above=0.1em] {нет} (bl41);
	\path [line] (bl3) -- node [left=0.2em] {да} (bl42);
	\path [line] (bl41) |- (bl2);
	\path [line] (bl42) -- (bl5);
\end{tikzpicture}
\end{center}
\begin{center}
\small{Рисунок 3}
\end{center}

\clearpage
\section{Программа на языке программирования C}

\lstinputlisting[language=c]{../program/main.c}

\subsection{Результаты программы}

\lstinputlisting[language=c]{../program/out.txt}

\section{Выводы}

Я научился решать нелинейные уравнения методом простых итераций,
методом Ньютона и модифицированным методом Ньютона с помощью ЭВМ.
\end{document}
