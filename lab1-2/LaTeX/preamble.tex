%%%
%%% -- author: https://github.com/Amet13/master-thesis
%%%
%%% Преамбула %%%


\usepackage{fontspec} % XeTeX
\usepackage{xunicode} % Unicode для XeTeX
\usepackage{xltxtra}  % Верхние и нижние индексы
\usepackage{pdfpages} % Вставка PDF
\usepackage{extsizes} % Для добавления в параметры класса документа 14pt
\usepackage{tikz}
\usepackage{pgfplots}
\pgfplotsset{compat=1.9}

\usepackage{listings} % Оформление исходного кода
\lstset{
    basicstyle=\footnotesize\ttfamily, % Размер и тип шрифта
    breaklines=true,                   % Перенос строк
    tabsize=4,                         % Размер табуляции
    literate={--}{{-{}-}}2,            % Корректно отображать двойной дефис
    literate={---}{{-{}-{}-}}3         % Корректно отображать тройной дефис
}

% Шрифты, xelatex
\defaultfontfeatures{Ligatures=TeX}
\setmainfont{Times New Roman} % Нормоконтроллеры хотят именно его
\newfontfamily\cyrillicfont{Times New Roman}
% \setsansfont{Liberation Sans} % Тут я его не использую, но если пригодится
\setmonofont{FreeMono} % Моноширинный шрифт для оформления кода

% Формулы
\usepackage{amsmath, amsfonts}
\usepackage{mathtools,unicode-math} % Не совместим с amsmath
\setmathfont{XITS Math}             % Шрифт для формул: https://github.com/khaledhosny/xits-math
%\numberwithin{equation}{section}    % Формула вида секция.номер

% Русский язык
\usepackage{polyglossia}
\setdefaultlanguage{russian}

% Абзацы и списки
\usepackage{enumerate}   % Тонкая настройка списков
\usepackage{indentfirst} % Красная строка после заголовка
\usepackage{float}       % Расширенное управление плавающими объектами
\usepackage{multirow}    % Сложные таблицы
\renewcommand{\baselinestretch}{1.5} % Полуторный межстрочный интервал
%\parindent 1.25cm % Абзацный отступ
\setlength\parindent{1.25cm}
% \setlength{\parskip}{0.5em}

\usepackage{textcase}

% Пути к каталогам с изображениями
\usepackage{graphicx} % Вставка картинок и дополнений
\graphicspath{{assets/}}

% Формат подрисуночных записей
\usepackage{chngcntr}

% Сбрасываем счетчик таблиц и рисунков в каждой новой главе
\counterwithin{figure}{section}
\counterwithin{table}{section}

% Гиперссылки
\usepackage{hyperref}
\hypersetup{
    colorlinks, urlcolor={black}, % Все ссылки черного цвета, кликабельные
    linkcolor={black}, citecolor={black}, filecolor={black},
}

% Оформление библиографии и подрисуночных записей через точку
\makeatletter
\renewcommand*{\@biblabel}[1]{\hfill#1.}
\renewcommand*\l@section{\@dottedtocline{1}{1em}{1em}}
\renewcommand{\thefigure}{\thesection.\arabic{figure}} % Формат рисунка секция.номер
\renewcommand{\thetable}{\thesection.\arabic{table}}   % Формат таблицы секция.номер
\def\redeflsection{\def\l@section{\@dottedtocline{1}{0em}{10em}}}
\makeatother

\sloppy             % Избавляемся от переполнений
\hyphenpenalty=1000 % Частота переносов
\clubpenalty=10000  % Запрещаем разрыв страницы после первой строки абзаца
\widowpenalty=10000 % Запрещаем разрыв страницы после последней строки абзаца

% Отступы у страниц
\usepackage[left=30mm,right=15mm,top=20mm,bottom=20mm]{geometry}

% Списки
\usepackage{enumitem}
\setlist[enumerate,itemize]{leftmargin=12.5mm} % Отступы в списках

\makeatletter
    \AddEnumerateCounter{\asbuk}{\@asbuk}{м)}
\makeatother
\setlist{nolistsep}                           % Нет отступов между пунктами списка
\renewcommand{\labelitemi}{—}                % Маркер списка —
% \renewcommand{\labelenumi}{\asbuk{enumi})}    % Список второго ровня
\renewcommand{\labelenumii}{\asbuk{enumii})} % Список третьего урвня

% Содержание
\usepackage{tocloft}
\renewcommand{\cfttoctitlefont}{\hspace{0.38\textwidth}\bfseries\MakeTextUppercase} % СОДЕРЖАНИЕ
\renewcommand{\cftsecfont}{\hspace{0pt}}            % Имена секций в содержании не жирным шрифтом
\renewcommand\cftsecleader{\cftdotfill{\cftdotsep}} % Точки для секций в содержании
\renewcommand\cftsecpagefont{\mdseries}             % Номера страниц не жирные
\setcounter{tocdepth}{1}                            % Глубина оглавления, до subsection

% Список иллюстративного материала
\renewcommand{\cftloftitlefont}{\hspace{0.17\textwidth}\MakeTextUppercase}
\renewcommand{\cftfigfont}{Рисунок }
\addto\captionsrussian{\renewcommand\listfigurename{Список иллюстративного материала}}

% Список табличного материала
\renewcommand{\cftlottitlefont}{\hspace{0.2\textwidth}\MakeTextUppercase}
\renewcommand{\cfttabfont}{Таблица }
\addto\captionsrussian{\renewcommand\listtablename{Список табличного материала}}

% Формат подрисуночных надписей
\RequirePackage{caption}
\DeclareCaptionLabelSeparator{defffis}{ -- } % Разделитель
\captionsetup[figure]{justification=centering, labelsep=defffis, format=plain} % Подпись рисунка по центру
\captionsetup[table]{justification=raggedright, labelsep=defffis, format=plain, singlelinecheck=false} % Подпись таблицы слева
\addto\captionsrussian{\renewcommand{\figurename}{Рисунок}} % Имя фигуры

% Пользовательские функции
\newcommand{\addimg}[4]{ % Добавление одного рисунка
    \begin{figure}
        \centering
        \includegraphics[width=#2\linewidth]{#1}
        \caption{#3} \label{#4}
    \end{figure}
}
\newcommand{\addimghere}[4]{ % Добавить рисунок непосредственно в это место
    \begin{figure}[H]
        \centering
        \includegraphics[width=#2\linewidth]{#1}
        \caption{#3} \label{#4}
    \end{figure}
}
\newcommand{\addtwoimghere}[5]{ % Вставка двух рисунков
    \begin{figure}[H]
        \centering
        \includegraphics[width=#2\linewidth]{#1}
        \hfill
        \includegraphics[width=#3\linewidth]{#2}
        \caption{#4} \label{#5}
    \end{figure}
}

% Секции без номеров (введение, заключение...), вместо section*{}
\newcommand{\anonsection}[1]{
    \phantomsection % Корректный переход по ссылкам в содержании
    \paragraph{\centerline{\MakeTextUppercase{{#1}}}\vspace{1em}}
    \addcontentsline{toc}{section}{#1}
}

% toc uppercase
\usepackage{etoolbox}
\makeatletter
\patchcmd{\l@section}{#1}{\MakeUppercase{#1}}{}{}
\patchcmd{\l@subsection}{#1}{\MakeUppercase{#1}}{}{}
\makeatother

% Секция для аннотации (она не включается в содержание)
\newcommand{\annotation}[1]{
    \paragraph{\centerline{{#1}}\vspace{1em}}
}

% Секция для списка иллюстративного материала
\newcommand{\lof}{
    \phantomsection
    \listoffigure
    \addcontentsline{toc}{section}{\listfigurename}
}

% Секция для списка табличного материала
\newcommand{\lot}{
    \phantomsection
    \listoftables
    \addcontentsline{toc}{section}{\listtablename}
}

% Секции для приложений
\newcommand{\appsection}[1]{
    \phantomsection
    \paragraph{\centerline{{#1}}}
    \addcontentsline{toc}{section}{{#1}}
}

% Библиография: отступы и межстрочный интервал
\makeatletter
\renewenvironment{thebibliography}[1]
    {\section*{\refname}
        \list{\@biblabel{\@arabic\c@enumiv}}
           {\settowidth\labelwidth{\@biblabel{#1}}
            \leftmargin\labelsep
            \itemindent 16.7mm
            \@openbib@code
            \usecounter{enumiv}
            \let\p@enumiv\@empty
            \renewcommand\theenumiv{\@arabic\c@enumiv}
        }
        \setlength{\itemsep}{0pt}
    }
\makeatother

\usepackage[nottoc,numbib]{tocbibind}	% добавить в toc и нумеровать

% Переопределение стандартных \section, \subsection, \subsubsection по ГОСТу;
% Переопределение их отступов до и после для 1.5 интервала во всем документе
\usepackage{titlesec}

\titleformat{\section}[block]
{\bfseries\normalsize}{\thesection}{1em}{\MakeTextUppercase}
\titlespacing\section{\parindent}{\parskip}{\parskip}

\titleformat{\subsection}[hang]
{\bfseries\normalsize}{\thesubsection}{1em}{}
\titlespacing\subsection{\parindent}{\parskip}{\parskip}

\titleformat{\subsubsection}[hang]
{\bfseries\normalsize}{\thesubsubsection}{1em}{}
\titlespacing\subsubsection{\parindent}{\parskip}{\parskip}


\usepackage{lastpage} % Подсчет количества страниц
\setcounter{page}{1}  % Начало нумерации страниц

\usetikzlibrary{shapes, arrows}
\tikzstyle{decision} = [diamond, draw, fill=white, 
    text width=4.5em, text badly centered, node distance=3cm, inner sep=0pt]
\tikzstyle{block} = [rectangle, draw, fill=white, 
    text width=12em, text centered, minimum height=2em]
\tikzstyle{line} = [draw, -latex']
\tikzstyle{cloud} = [draw, ellipse,fill=red!20, node distance=3cm,
    minimum height=2em]
